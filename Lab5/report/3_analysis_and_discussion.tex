\subsection{Training Curves}

Task 1:

\begin{figure}[H]
    \centering
    \includegraphics[width=0.8\textwidth]{src/task1.png}
    \caption{Training curves for Task 1}
    \label{fig:task1_training_curves}
\end{figure}

Task 2:

\begin{figure}[H]
    \centering
    \includegraphics[width=0.8\textwidth]{src/task2.png}
    \caption{Training curves for Task 2}
    \label{fig:task2_training_curves}
\end{figure}

Task 3:

\begin{figure}[H]
    \centering
    \includegraphics[width=0.8\textwidth]{src/task3.png}
    \caption{Training curves for Task 3}
    \label{fig:task3_training_curves}
\end{figure}


\subsection{Sample Efficiency}

It is obvious that the orange line (PER) is more sample efficient than the green line (without PER).

\begin{figure}[H]
    \centering
    \includegraphics[width=0.8\textwidth]{src/task3-per.png}
    \caption{Sample efficiency comparison}
    \label{fig:sample_efficiency}
\end{figure}
