% Include:
% - How to run the code
% - Any additional instructions

\subsection{How to run the code}

Before running the code, make sure you are in the right directory.

\lstinline|$ tree|

\begin{verbatim}
├── dataset
├── saved_models
└── src
\end{verbatim}

You can run the code by executing the following command:

First, install the dependencies:

\begin{lstlisting}[language=bash, caption=Install the dependencies, label=lst:install_dependencies]
pip install -r requirements.txt
\end{lstlisting}

Before running the code, you can disable wandb to prevent some issues.

\begin{lstlisting}[language=bash, caption=Disable wandb, label=lst:disable_wandb]
wandb disabled
\end{lstlisting}

To re-enable wandb, you can run the following command:

\begin{lstlisting}[language=bash, caption=Enable wandb, label=lst:enable_wandb]
wandb enabled
\end{lstlisting}

Then, run the training code:

\begin{lstlisting}[language=bash, caption=How to run the training code, label=lst:run_train_code]
# Run the code for training
python src/train.py --data_path dataset/oxford-iiit-pet/
\end{lstlisting}

there are other parameters you can use to customize the training process:

\begin{itemize}
    \item \lstinline|--device|: the device to run the code on (default: cuda:1)
    \item \lstinline|--seed|: the seed for the random number generator (default: 88)
    \item \lstinline|--epochs|: the number of epochs to train the model (default: 500)
    \item \lstinline|--batch_size|: the batch size for the training process (default: 64)
    \item \lstinline|--learning_rate|: the learning rate for the training process (default: 7e-4)
    \item \lstinline|--model|: the model to use for the training process (default: resnet34\_unet), you can choose from: resnet34\_unet, unet
    \item \lstinline|--wandb|: whether to use wandb for logging (default: True)
\end{itemize}

For my own results, I used the following commands:

\begin{lstlisting}[language=bash, caption=How to run the training code for my own results, label=lst:run_train_code_my_results]
# ResNet34 UNet
python src/train.py \
    --data_path dataset/oxford-iiit-pet/ \
    -d cuda:1 \
    -s 228922

# UNet
python src/train.py \
    --data_path dataset/oxford-iiit-pet/ \
    -d cuda:1 \
    -s 228922 \
    -m unet
\end{lstlisting}

After training, you can run the inference code:

\begin{lstlisting}[language=bash, caption=How to run the inference code, label=lst:run_inference_code]
# Run the code for inference
# Replace the model_path with the path to the model you want to use for the inference
# Model name should contain resnet34_unet or unet
python src/inference.py --data_path dataset/oxford-iiit-pet/ --model_path saved_models/model.pth
\end{lstlisting}
