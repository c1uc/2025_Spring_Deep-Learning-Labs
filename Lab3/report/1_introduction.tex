In this lab, we will implement the MaskGIT for image inpainting.
There are three stages in this lab: Train a VQVAE to encode the image into a discrete latent space, train a MaskGIT to generate the image from masked latent codes, and use the MaskGIT to inpaint the image.

\subsection{VQVAE}
    The VQVAE is a type of variational autoencoder that uses a discrete latent space.
    The encoder maps the input image to a latent code, and the decoder maps the latent code to the reconstructed image.
    The latent code is a discrete codebook, which is a set of discrete latent codes.
    The decoder maps the latent code to the reconstructed image.

\subsection{MaskGIT}
    MaskGIT is a type of generative adversarial network that uses a masked latent code to generate the image.
    In training stage, we use the VQVAE to encode the image into a latent code, and then use the MaskGIT to generate the image from the masked latent code.

\subsection{Image Inpainting}
    In the image inpainting stage, we use the MaskGIT to inpaint some masked images.
    For each iteration, we paint all the masked pixels with the generated image from the MaskGIT.
    Then, we choose specified percentage of the regions where the agent has less confidence to paint, and mask them again.
    The percentage of each iteration is configured by the mask scheduling function.