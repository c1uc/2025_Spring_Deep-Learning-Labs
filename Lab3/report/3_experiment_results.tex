\subsection{Iterative Decoding}

For iterative decoding, we implemented three different mask scheduling functions: linear, cosine, and square.
There scheduling curve is shown in Figure \ref{fig:mask_scheduling_curve}.

\begin{figure}[hb]
    \centering
    \includegraphics[width=0.5\textwidth]{src/gamma_funcs.png}
    \caption{Mask Scheduling Curve}
    \label{fig:mask_scheduling_curve}
\end{figure}

For their behavior in the iterative decoding, we can see the comparison in Figure \ref{fig:mask_scheduling_comparison}.
From the figure, we can see that the linear scheduling function decreases faster than the square and cosine scheduling functions, and cosine and square scheduling functions have similar performance.

\begin{table}[hb]
    \centering
    \begin{tabular}{c}
        Linear
        \includegraphics[width=0.6\textwidth]{src/mask_linear.png} \\
        \hline
        Square
        \includegraphics[width=0.6\textwidth]{src/mask_square.png} \\
        \hline
        Cosine
        \includegraphics[width=0.6\textwidth]{src/mask_cosine.png}
    \end{tabular}
    \caption{Comparison of different mask scheduling functions in iterative decoding}
    \label{fig:mask_scheduling_comparison}
\end{table}

And the following figure \ref{fig:inpainting_comparison} is the comparison of the iterative decoding procedure.
Their performance is really close, and cosine and square has slightly better performance than linear. They all have FID scores around 31 afrer 20 iterations of decoding.

\begin{table}[hb]
    \centering
    \begin{tabular}{c}
        Linear
        \includegraphics[width=0.6\textwidth]{src/inpaint_linear.png} \\
        \hline
        Square
        \includegraphics[width=0.6\textwidth]{src/inpaint_square.png} \\
        \hline
        Cosine
        \includegraphics[width=0.6\textwidth]{src/inpaint_cosine.png}
    \end{tabular}
    \caption{Comparison of different mask scheduling functions in inpainting}
    \label{fig:inpainting_comparison}
\end{table}

\subsection{Visualization}

For the best performance decoding, I found that using square or cosine scheduling with sweet spot at 1 epoch has the best performance.
And the following figure \ref{fig:inpainting_visualization} is the visualization of the decoding results.

\begin{table}[hb]
    \centering
    \begin{tabular}{c|ccccc}
        & Image 0 & Image 1 & Image 2 & Image 3 & Image 4 \\
        \hline
        Masked & 
        \includegraphics[width=0.15\textwidth]{src/image_000_masked.png} &
        \includegraphics[width=0.15\textwidth]{src/image_001_masked.png} &
        \includegraphics[width=0.15\textwidth]{src/image_002_masked.png} &
        \includegraphics[width=0.15\textwidth]{src/image_003_masked.png} &
        \includegraphics[width=0.15\textwidth]{src/image_004_masked.png} \\
        Decoded &
        \includegraphics[width=0.15\textwidth]{src/image_000.png} &
        \includegraphics[width=0.15\textwidth]{src/image_001.png} &
        \includegraphics[width=0.15\textwidth]{src/image_002.png} &
        \includegraphics[width=0.15\textwidth]{src/image_003.png} &
        \includegraphics[width=0.15\textwidth]{src/image_004.png}
    \end{tabular}
    \caption{Visualization of inpainting results. Rows show masked input images and their corresponding decoded outputs.}
    \label{fig:inpainting_visualization}
\end{table}

\subsection{FID Score}
The result is shown in the screenshot \ref{fig:fid_score} below, the best FID score is 30.71.

\begin{figure}[hb]
    \centering
    \includegraphics[width=0.8\textwidth]{src/fid_score.png}
    \caption{FID Score}
    \label{fig:fid_score}
\end{figure}
